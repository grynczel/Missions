\documentclass[a4paper]{article}
\usepackage[utf8]{inputenc}
\usepackage[T1]{fontenc}
\usepackage[pdftex]{graphicx}
\usepackage{fancyhdr}
\usepackage{lscape}
\usepackage{color}
\usepackage[english]{babel}
\usepackage{graphicx}
\usepackage[colorinlistoftodos]{todonotes}
\usepackage{listings}
\usepackage{color}
\usepackage{changepage}
\usepackage[margin=1in]{geometry}
\definecolor{codegreen}{rgb}{0,0.6,0}
\definecolor{codegray}{rgb}{0.5,0.5,0.5}
\definecolor{codepurple}{rgb}{0.58,0,0.82}
\definecolor{backcolour}{rgb}{0.95,0.95,0.92}
 
 \lstdefinestyle{mystyle}{
 	backgroundcolor=\color{backcolour},   
 	commentstyle=\color{codegreen},
 	keywordstyle=\color{magenta},
 	numberstyle=\tiny\color{codegray},
 	stringstyle=\color{codepurple},
 	basicstyle=\footnotesize,
 	breakatwhitespace=false,         
 	breaklines=true,                 
 	captionpos=b,                    
 	keepspaces=true,                 
 	numbers=left,                    
 	numbersep=5pt,                  
 	showspaces=false,                
 	showstringspaces=false,
 	showtabs=false,                  
 	tabsize=2
 }
 
\lstset{
	style=mystyle,
	inputencoding=utf8,
	extendedchars=true,
	literate={á}{{\'a}}1 {ã}{{\~a}}1 {é}{{\'e}}1,
	escapechar=\&
}
\title{Algorithmique et structures de données : Mission 1 (produit)}
\date{03 octobre 2014}
\author{Ivan Ahad - Jérôme Bertaux - Rodolphe Cambier - Guillaume Coutisse\\ 
	Baptiste Degryse - Wojciech Grynczel - Joachim De Droogh - Thomas Grimée - Benoît Ickx}

\begin{document}	
\maketitle

\section*{Commentaire sur la solution apportée}

Il nous a été demandé de concevoir et d'implémenter un interpréteur du mini langage PostScript. L’interpréteur prendra en entrée un fichier contenant les opérations à effectuer et produira en sortie un nouveau fichier résultant des opérations demandées.
Pour réaliser cela, il nous a fallu lire le fichier, savoir lire et décrypter les informations s'y trouvant, générer un fichier de sortie et stocker les valeurs à écrire dans celui-ci.

\subsubsection*{Lire le fichier}

Pour lire le fichier, nous avons utilisé les classes FileReader et BufferedReader. Grâce à elles, nous avons accès aux lignes d'opérations que nous stockons dans un arraylist. Cet arraylist contiendra toutes les lignes à exécuter via notre interpréteur.

\subsubsection*{Lire et décrypter les informations}

Chaque ligne va être interprétée. Les mots (token dans le programme) seront comparés aux expressions connues telles "add", "sub", etc. 
\begin{itemize}
\item Si l'expression n'est pas connue, c'est que le token correspond à un nombre ou à une variable prédéfinie. Ce nombre/variable sera stocké dans la \textit{Stack}. 

\begin{lstlisting}[language=Java]
expressionStack.push(new Number(token));
expressionStack.push(new Variable(token));	
\end{lstlisting}
		
\item Si l'expression est connue, on l'effectue telle que voulu. Par exemple pour les opérateurs, on utilise les deux éléments au sommet de la \textit{Stack} afin d'effectuer l'opération. L'élément de la \textit{Stack} qui est passé est interprété afin que si il représente une variable, il soit remplacé par sa valeur enregistrée dans une \textit{Map}. Ces deux éléments sont alors enlevés da la \textit{Stack} et remplacés par le résultat de l'opération.

 
 \begin{lstlisting}[language=Java]
if (token.equals("add")) {
	Expression first = expressionStack.pop();
	Expression second = expressionStack.pop();
	Expression subExpression = new Add(first, second);
	expressionStack.push(new Number(subExpression.interpret(variables)));
} 
else if (token.equals("sub")) {
				[...]
} 
else if (token.equals("mul")) {
				[...]
}
\end{lstlisting}

\end{itemize}
\subsubsection*{Générer le fichier de sortie}
%A chaque appel de "pstack" nous devons écrire l'élément se trouvant au sommet de la pile dans le fichier de sortie. Pour effectuer cela, nous utilisons une arraylist contenant les données à écrire. Cette arraylist va donc contenir toute la sortie. Pour écrire ces données, nous faisons appel à la classe PrintWriter. ce qui nous permet de créer le fichier de sortie contenant toutes les données qu'on devait afficher.
A chaque fois que le token \textit{pstack} est rencontré, le contenu de la \textit{Stack} est imprimé grâce à un \textit{PrintWriter} dans un fichier texte.

\end{document}