\documentclass[a4paper]{article}
\usepackage[utf8]{inputenc}
\usepackage[T1]{fontenc}
\usepackage[pdftex]{graphicx}
\usepackage{fancyhdr}
\usepackage{lscape}
\usepackage{color}
\usepackage{qtree}
\usepackage[english]{babel}
\usepackage{graphicx}
\usepackage[colorinlistoftodos]{todonotes}
\usepackage{listings}
\usepackage{color}
\usepackage{changepage}
\usepackage[margin=1in]{geometry}
\definecolor{codegreen}{rgb}{0,0.6,0}
\definecolor{codegray}{rgb}{0.5,0.5,0.5}
\definecolor{codepurple}{rgb}{0.58,0,0.82}
\definecolor{backcolour}{rgb}{0.95,0.95,0.92}
\usepackage[parfill]{parskip}

 \lstdefinestyle{mystyle}{
 	backgroundcolor=\color{backcolour},   
 	commentstyle=\color{codegreen},
 	keywordstyle=\color{magenta},
 	numberstyle=\tiny\color{codegray},
 	stringstyle=\color{codepurple},
 	basicstyle=\footnotesize,
 	breakatwhitespace=false,         
 	breaklines=true,                 
 	captionpos=b,                    
 	keepspaces=true,                 
 	numbers=left,                    
 	numbersep=5pt,                  
 	showspaces=false,                
 	showstringspaces=false,
 	showtabs=false,                  
 	tabsize=2
 }
 
\lstset{
	style=mystyle,
	inputencoding=utf8,
	extendedchars=true,
	literate={á}{{\'a}}1 {ã}{{\~a}}1 {é}{{\'e}}1,
	escapechar=\&
}
\title{Algorithmique et structures de données : Mission 3}
\date{24 octobre 2014}
\author{Groupe 1.2: Ivan Ahad - Jérôme Bertaux - Rodolphe Cambier \\ 
	Baptiste Degryse - Wojciech Grynczel - Charles Jaquet}



\begin{document}
\maketitle

\paragraph*{Question 1 (Charles Jacquet)}
\begin{itemize}
\item{\textbf{Les clés doivent-elles automatiquement être des nombres}}\\
Non, elles peuvent être n'importe quoi tant que c'est comparable.
Par exemple, ça pourrait être des String classé de manière alphabétique.
\item{\textbf{Enumérer en ordre croissant toute les clés mémorisées}}\\
il suffit d'utiliser une fonction récursive, qui va se réappeller à chaquer élément de telle sorte que :
String s = recursiveFunction(tree);

avec comme pseudo code
\begin{verbatim}
public String recursiveFunction(BinaryTree tree){
		if (tree.left == null && tree.right == null){
			return tree.getElem();
		}
		else if(tree.left == null){
			return recursiveFunction(tree.right);
		}
		else if(tree.right == null){
			return recursiveFunction(tree.left);
		}
		else{
			return recursiveFunction(tree.left) + recursiveFunction(tree.rigth);
		}
}
\end{verbatim}
La complexité de cette méthode est en O(h) avec h la hauteur du root.

\item{\textbf{Dans le cas où une clé est mémorisée deux fois}}\\
Lors de la deuxième mémorisation, dans le livre il est marqué qu'elle remplace la première. Il n'y a donc pas de relation père-fils.

\end{itemize}
\paragraph*{Question 2}
\paragraph*{Question 3}
\paragraph*{Question 4}
\paragraph*{Question 5}
\paragraph*{Question 6}
\paragraph*{Question 7}
\paragraph*{Question 8 (Baptiste Degryse)}
La propriété la moins évidente à vérifier est l'équilibre de l'arbre. Puisqu'il y a 1000 éléments, la profondeur de l'arbre doit être de maximum $log_2(1000) = 9.9 \simeq 10 $. Donc, pour la racine, il doit y avoir entre 512 et 488 clés dans chaque sous-arbre si toutes les clés sont utilisées. Sinon, il faut avoir un minimum de $2^{n-1}$ où n est la hauteur visible lors de la recherche.

2, 252, 401, 398, 330, 344, 397, 363 : Impossible, l'arbre n'est pas équilibré ( donc pas AVL ) car il y a maximum 1 enfant à gauche de la racine. (le chiffre 1)

924, 220, 911, 244, 898, 258, 362, 363 : $2^{7-1} = 64$. Il n'y a donc pas de problème pour celui-ci, mais il y a une grande densité de présence de clé pour les clés élevées.

925, 202, 911, 240, 912, 245, 363 : idem

2, 399, 387, 219, 266, 382, 381, 278, 363 : Impossible, l'arbre n'est pas équilibré ( donc pas AVL ) car il y a maximum 1 enfant à gauche de la racine. (le chiffre 1)

935, 278, 347, 621, 299, 392, 358, 363 : il faut 64 éléments dans l'autre sous arbre, ce qui implique que toutes les clés sauf une doivent être présentes au dessus de 934 pour que l'arbre soit valide.


\end{document}
